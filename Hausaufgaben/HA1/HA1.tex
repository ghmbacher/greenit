\documentclass[ngerman]{article}
\usepackage{graphicx} % Für Bilder
\usepackage{adjustbox} % Für Rotationen und Skalierungen
\usepackage[a4paper,margin=2.5cm]{geometry} % Ränder setzen
\usepackage[ngerman]{babel} % Deutsche Spracheinstellungen
\usepackage{csquotes} % Korrekte Anführungszeichen
\usepackage{datetime} % Für deutsches Datum
\renewcommand{\dateseparator}{.} % Punkt als Datumstrenner

\begin{document}

\title{Hausübung 1; UV WAP WS 2024/25}
\author{Gerwin Bacher \\
MatrNr: 12314104}
\date{\today}

\maketitle

\begin{table}[h!]
\begin{tabular}{|c|c|c|}
\hline
\textbf{Originalgröße} & \textbf{Skaliert (50\%)} & \textbf{Rotiert (45°)} \\
\hline
\includegraphics[width=0.25\textwidth]{image.png} &
\includegraphics[width=0.125\textwidth]{image.png} &
\adjustbox{rotate=45}{\includegraphics[width=0.25\textwidth]{image.png}} \\
\hline
\textbf{Skaliert (150\%)} & \textbf{Skaliert + + Rotiert (30°)} & \textbf{Kleine Größe + Rotiert (90°)} \\
\hline
\includegraphics[width=0.375\textwidth]{image.png} &
\adjustbox{rotate=30}{\includegraphics[width=0.125\textwidth]{image.png}} &
\adjustbox{rotate=90}{\includegraphics[width=0.075\textwidth]{image.png}} \\
\hline
\end{tabular}
\caption{Tabelle mit unterschiedlich skalierten und rotierten Bildern}
\end{table}

\end{document}
